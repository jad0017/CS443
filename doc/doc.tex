\documentclass{article}
\usepackage{fullpage}
\usepackage{setspace}
\usepackage{changepage}
\usepackage{listings}
\usepackage{amsmath}
\usepackage[perpage]{footmisc}
\usepackage[normalem]{ulem}
\usepackage{graphicx}
\usepackage{float}
\graphicspath{ {images/} }
\begin{document}

\lstset{
	escapeinside={(*@}{@*)},
	tabsize=4,
	frame=single,
	columns=flexible,
	breaklines=true
}


\newcommand{\DropImageTwo}[2][]{%
	\begin{figure}[H]
		\centering
			\fbox{\includegraphics[height=\textwidth,width=\textwidth,keepaspectratio]{#2}}
		\caption{#1}
	\end{figure}
}


\newcommand{\DropImage}[2][]{%
	\begin{figure}[H]
		\centering
			\fbox{\includegraphics[height=\textwidth,width=\textwidth,keepaspectratio]{#2}}
		\caption{#1}
	\end{figure}
}

\begin{center}
{\Huge\textbf{Group Project 1}} \\
{\Large Team Sparkle Motion:} \\
{Joshua Dickson\hspace*{2ex}Christy Lafayette\hspace*{2ex}Paul Eccleston}
\end{center}
\vspace*{4ex}


\section{How To Run}

Note: This project work with color (RGB) and grayscale images.

\noindent
Compressing an image:
\begin{lstlisting}
python3 ./main.py --compress --dump-dct-coefficients=out.dct in.jpg out.ncage
\end{lstlisting}

\noindent
\texttt{--chroma-subsampling} is also a valid option.  See \texttt{--help} or
README for more information.


\noindent
Decompressing an Image:
\begin{lstlisting}
python3 ./main.py --decompress in.ncage out.bmp
\end{lstlisting}


\noindent
Computing difference/error:
\begin{lstlisting}
python3 ./error.py orig.jpg new.bmp
\end{lstlisting}


\noindent
Computing difference (Image Form):
\begin{lstlisting}
python3 ./diff.py orig.jpg new.bmp out.diff.bmp
\end{lstlisting}


\noindent
Dumping DCT coefficient block data:
\begin{lstlisting}
python3 ./dct.py --by-block in.dct 12 12
\end{lstlisting}

\noindent
See \text{--help} text for more information and other options.


\section{Output}

\subsection{Using 4:4:4 subsampling}
\begin{lstlisting}
python3 ./main.py --compress in.jpg out.ncage
python3 ./main.py --decompress out.ncage out.bmp
python3 ./diff.py in.jpg out.bmp diff.bmp
\end{lstlisting}


\DropImage[Barn: Original]{barn}
\DropImage[Barn: 4:4:4 Compressed]{barn-out}
\DropImage[Barn: Diff 4:4:4 Compressed]{barn-diff}


\DropImage[Sloth: Original]{sloth}
\DropImage[Sloth: 4:4:4 Compressed]{sloth-out}
\DropImage[Sloth: Diff 4:4:4 Compressed]{sloth-diff}

\DropImage[Nick Cage: Original]{ncage}
\DropImage[Nick Cage: 4:4:4 Compressed]{ncage-out}
\DropImage[Nick Cage: Diff 4:4:4 Compressed]{ncage-diff}


\subsection{DCT Output}

\begin{lstlisting}
python3 ./main.py --compress --dump-dct-coefficients=barn.dct barn.jpg barn.ncage
python3 ./dct.py --print-size barn.dct
\end{lstlisting}

\noindent Output:
\lstinputlisting{size}

\newpage
\begin{lstlisting}
python3 ./dct.py --by-block barn.dct 12 12
\end{lstlisting}

\noindent Output:
\lstinputlisting{block}

\begin{lstlisting}
python3 ./dct.py --by-block-num barn.dct 12
\end{lstlisting}

\noindent Output:
\lstinputlisting{block-num}

\newpage
\begin{lstlisting}
python3 ./dct.py --by-pixel barn.dct 12
\end{lstlisting}

\noindent Output:
\lstinputlisting{pixel}

\subsection{Error Computation}

\begin{lstlisting}
python3 ./main.py --compress barn.jpg barn.ncage
python3 ./main.py --decompress barn.ncage barn.bmp
python3 ./error.py barn.jpg barn.bmp
\end{lstlisting}

\noindent Output:
\lstinputlisting{barn-error}

\newpage
\begin{lstlisting}
python3 ./main.py --compress sloth.jpg sloth.ncage
python3 ./main.py --decompress sloth.ncage sloth.bmp
python3 ./error.py sloth.jpg sloth.bmp
\end{lstlisting}

\noindent Output:
\lstinputlisting{sloth-error}

\begin{lstlisting}
python3 ./main.py --compress ncage.jpg ncage.ncage
python3 ./main.py --decompress ncage.ncage ncage.bmp
python3 ./error.py ncage.jpg ncage.bmp
\end{lstlisting}

\noindent Output:
\lstinputlisting{ncage-error}




\subsection{Chroma Subsampling Differences}

\begin{lstlisting}
python3 ./main.py --compress --chroma-subsampling=4:2:2 sloth.jpg sloth.422.ncage
python3 ./main.py --decompress sloth.422.ncage sloth.422.bmp
python3 ./main.py --compress --chroma-subsampling=4:2:0 sloth.jpg sloth.420.ncage
python3 ./main.py --decompress sloth.420.ncage sloth.420.bmp
python3 ./main.py --compress --chroma-subsampling=4:1:1 sloth.jpg sloth.411.ncage
python3 ./main.py --decompress sloth.411.ncage sloth.411.bmp
python3 ./main.py --compress --chroma-subsampling=4:1:0 sloth.jpg sloth.410.ncage
python3 ./main.py --decompress sloth.410.ncage sloth.410.bmp
\end{lstlisting}

\newpage
\begin{lstlisting}
python3 ./diff.py sloth.422.bmp sloth.420.bmp sloth.422-420.bmp
\end{lstlisting}

\DropImageTwo[Sloth: Diff 4:2:2 and 4:2:0 Chroma Subsampling]{sloth-422-420-diff}

\newpage
\begin{lstlisting}
python3 ./diff.py sloth.422.bmp sloth.411.bmp sloth.422-411.bmp
\end{lstlisting}

\DropImageTwo[Sloth: Diff 4:2:2 and 4:1:1 Chroma Subsampling]{sloth-422-411-diff}

\newpage
\begin{lstlisting}
python3 ./diff.py sloth.422.bmp sloth.410.bmp sloth.422-410.bmp
\end{lstlisting}

\DropImageTwo[Sloth: Diff 4:2:2 and 4:1:0 Chroma Subsampling]{sloth-422-410-diff}



\section{Pre-Meeting Notes (by Josh)}

\subsection{Overview}

This document is a quick layout of the compression and decompression
streams according to the assignment specification.

\subsection{Compression}

The compression routine follows the following steps:
\begin{enumerate}
\item Convert RGB to $\text{YC}_{b}\text{C}_{r}$
\item Perform chroma subsampling if requested
\item Apply 2D DCT transform on Y, $\text{C}_{b}$, and $\text{C}_{r}$ components
\item Apply Quantization
\item Apply Run-Length Coding (RLC)
\end{enumerate}

\subsubsection{RGB to $\text{YC}_{b}\text{C}_{r}$}

Simple conversion.  We did this for assignment 2.  I used a modification to
the conversion matrix (shift values of 128 in place of the 0.5s).
Alternate to my alternate would be shifting by 127.
We need all values to be within the integer range $[0,256)$
for the DCT calculation.

Original:
\[
\begin{bmatrix}
Y \\
C_{b} \\
C_{r}
\end{bmatrix}
=
\begin{bmatrix}
0.299 & 0.587 & 0.144 \\
-0.168736 & -0.331264 & 0.5 \\
0.5 & -0418688 & -0.081312
\end{bmatrix}
\begin{bmatrix}
R \\
G \\
B
\end{bmatrix}
+
\begin{bmatrix}
0\\
0.5\\
0.5
\end{bmatrix}
\]


Modification:
\[
\begin{bmatrix}
Y \\
C_{b} \\
C_{r}
\end{bmatrix}
=
\begin{bmatrix}
0.299 & 0.587 & 0.144 \\
-0.168736 & -0.331264 & 0.5 \\
0.5 & -0418688 & -0.081312
\end{bmatrix}
\begin{bmatrix}
R \\
G \\
B
\end{bmatrix}
+
\begin{bmatrix}
0\\
128\\
128
\end{bmatrix}
\]


\subsubsection{Chroma Subsampling}

This is an optional step.  The user can request any chroma
subsampling value.  I suggest restricting this to a high-order
of 4 so we have a small subset for easy input validation.
The spec would be as follows:
\begin{itemize}
	\item $A$:$X$:$Y$ is invalid for $A \neq 4$
	\item 4:4:4 means no change
	\item 4:0:$X$ means duplicate the second row
	\item 4:$X$:0 means duplicate the first row
	\item 4:0:0 is invalid
	\item 4:3:$X$ and 4:$X$:3 are invalid
	\item 4:$A$:$B$ is invalid for $A$ or $B > 4$
	\item A component of '1' means select the first pixel only
	\item A component of '2' means select the first and third pixels only
\end{itemize}

\subsubsection{2D DCT Transform}

We will be using 8 x 8 blocks.
The following equations operate on 8 x 8 matrices.

Equation 1.1:
\[
D(i,j) = \frac{1}{\sqrt{2N}} C(i) C(j)
\sum\limits_{x=0}^{N-1}
\sum\limits_{y=0}^{N-1}
p(x,y)
\text{cos}\bigg{[}
\frac{i\pi(2x + 1)}{2N}
\bigg{]}
\text{cos}\bigg{[}
\frac{j\pi(2y + 1)}{2N}
\bigg{]}
\]

Equation 1.2:
\[
C(u) =
\begin{cases}
\frac{1}{\sqrt{2}} & \text{if\ } u = 0 \\
1 & \text{if\ } u > 0
\end{cases}
\]

Variables:
\[
p(x,y) = \text{the (x,y)-th element of the image represented by matrix p}
\]
\[
N = \text{the size of the DCT block eight (8) in our case.}
\]

Simplified Equation 1.1 for 8 x 8 blocks:
\[
D(i,j) = \frac{1}{4} C(i) C(j)
\sum\limits_{x=0}^{7}
\sum\limits_{y=0}^{7}
p(x,y)
\text{cos}\bigg{[}
\frac{i\pi(2x + 1)}{16}
\bigg{]}
\text{cos}\bigg{[}
\frac{j\pi(2y + 1)}{16}
\bigg{]}
\]

The derivative Translation matrix from Equation 1.1 is found with:
\[
T(i,j) =
\begin{cases}
\frac{1}{\sqrt{N}} & \text{if\ } i = 0 \\
\sqrt{\frac{2}{N}} \text{cos}\bigg{[}\frac{i\pi(2j+1)}{2N}\bigg{]} & \text{if\ } i > 0
\end{cases}
\]

We also need the transpose of $T$: $T'$.  This thing:
\[
\begin{bmatrix}
1 & 2 & 3 \\
4 & 5 & 6
\end{bmatrix}^{T}
=
\begin{bmatrix}
1 & 4 \\
2 & 5 \\
3 & 6
\end{bmatrix}
\]

Once those are found, the DCT2 coefficient block is simply:
\[ D = T M T' \]

That wasn't so hard, right?

\subsubsection{Quantization}

This stage includes the following steps for each component ($\text{Y, C}_{b}\text{, C}_{r}$):
\begin{enumerate}
	\item Scale the base quantization matrix ($B$) to the quality factor given
	\item Quantize the input DCT Coefficient matrix ($D$) with the scaled quantization matrix ($Q$)
\end{enumerate}

We were given two base quantization matrices:

\[B_{luminance} =
\begin{bmatrix}
16 & 11 & 10 & 16 & 24 & 40 & 51 & 61 \\
12 & 12 & 14 & 19 & 26 & 58 & 60 & 55 \\
14 & 13 & 16 & 24 & 40 & 57 & 69 & 56 \\
14 & 17 & 22 & 29 & 51 & 87 & 80 & 62 \\
18 & 22 & 37 & 56 & 68 & 109 & 113 & 77 \\
24 & 35 & 55 & 64 & 81 & 104 & 113 & 92 \\
49 & 64 & 78 & 87 & 103 & 121 & 120 & 101 \\
72 & 92 & 95 & 98 & 112 & 100 & 103 & 99
\end{bmatrix}
\]

\[B_{chromanance} =
\begin{bmatrix}
17 & 18 & 24 & 47 & 99 & 99 & 99 & 99 \\
18 & 21 & 26 & 66 & 99 & 99 & 99 & 99 \\
24 & 26 & 56 & 99 & 99 & 99 & 99 & 99 \\
47 & 66 & 99 & 99 & 99 & 99 & 99 & 99 \\
99 & 99 & 99 & 99 & 99 & 99 & 99 & 99 \\
99 & 99 & 99 & 99 & 99 & 99 & 99 & 99 \\
99 & 99 & 99 & 99 & 99 & 99 & 99 & 99 \\
99 & 99 & 99 & 99 & 99 & 99 & 99 & 99
\end{bmatrix}
\]

\subsubsection{Scale the Base Quantization Matrix}

The scale factor (where $s_{f}$ is the scale factor and $q$ is the requested quality):
\[
s_{f} =
\begin{cases}
\frac{100 - q}{50} & \text{if\ } q >= 50 \\
\frac{50}{q} & \text{otherwise}
\end{cases}
\]

\noindent
The resultant, scaled quantization matrix $Q$ (where $B$ is the base quantization matrix):
\[
Q_{i,j} =
\begin{cases}
\text{clip}(\text{round}(B_{i,j} * s_{f})) & \text{if\ } s_{f} \ne 0 \\
B_{i,j} & \text{if\ } s_{f} = 0
\end{cases}
\]
\noindent
The clip function:
\[
\text{clip}(x) =
\begin{cases}
255 & \text{if\ } x > 255 \\
x & \text{otherwise}
\end{cases}
\]

\subsubsection{Quantize With Scaled Matrix}

The equation for the resultant quantized input $C$ (where $D$ is the DCT Coefficient Matrix
and $Q$ is the scaled quantization matrix):
\[
C_{i,j} = \text{round}\bigg{(}\frac{D_{i,j}}{Q_{i,j}}\bigg{)}
\]

\subsubsection{Run-Length Coding (RLC)}

Now that we have the quantized matrix $C$, we need to apply Run-Length Coding
in a \textit{snake} pattern.  I decided to just write up the algorithm for this
because it's annoying.  This takes advantage of a constant translation array
for 8 x 8 matrices as follows.
\newpage
\begin{lstlisting}[language=python]
# RLC translational lookup array
R = [
    (0,0),
    (0,1), (1,0),
    (2,0), (1,1), (0,2),
    (0,3), (1,2), (2,1), (3,0),
    (4,0), (3,1), (2,2), (1,3), (0,4),
    (0,5), (1,4), (2,3), (3,2), (4,1), (5,0),
    (6,0), (5,1), (4,2), (3,3), (2,4), (1,5), (0,6),
    (0,7), (1,6), (2,5), (3,4), (4,3), (5,2), (6,1), (7,0),
    (1,7), (2,6), (3,5), (4,4), (5,3), (6,2), (7,1),
    (7,2), (6,3), (5,4), (4,5), (3,6), (2,7),
    (3,7), (4,6), (5,5), (6,4), (7,3),
    (7,4), (6,5), (5,6), (4,7),
    (5,7), (6,6), (7,5),
    (7,6), (6,7),
	(7,7)
]

# Output is an array of tuples of the form (count, value).
def RLC(C):
	global R
	a = []
	current = C[0][0]
	count = 0
	for i in range(len(R)):
		(y,x) = R[i]
		if C[y][x] == current:
			count += 1
		else:
			a.append( (count, current) )
			current = C[y][x]
			count = 1
	a.append( (count, current) )
	return a
\end{lstlisting}

\subsection{Decompression}

The decompression routine follows the following steps:
\begin{enumerate}
\item Apply Inverse Run-Length Coding (iRLC)
\item Apply De-Quantization
\item Apply the Inverse 2D DCT transform (iDCT)
\item Convert $\text{YC}_{b}\text{C}_{r}$ to RGB
\end{enumerate}

\noindent
Note I've skipped his 3: Get the $\text{YC}_{b}\text{C}_{r}$
values for each pixel by assigning values from subsampled chroma data.
It was skipped because it didn't make sense\ldots\ superfluous even.

\subsubsection{Inverse Run-Length Coding (iRLC)}

This is simple again, I just opted to write it too.
This function takes the tuple array form from the RLC
algorithm and also uses the same translation array ($R$).

\begin{lstlisting}[language=python]
# Create a zero matrix of size n x m
def zero_matrix(n, m):
	return [ [ 0 for c in range(n) ] for r in range(m) ]

def iRLC(a):
	global R
	n = math.sqrt(len(R)) # or just 8...
	A = zero_matrix(n, n)
	idx = 0
	for i in range(len(a)):
		(count, val) = a[i]
		for j in range(count):
			(y,x) = R[idx]
			idx += 1
			A[y][x] = val
	return A
\end{lstlisting}

\subsubsection{De-Quantization}

We use the same scaling algorithm from Section 2.4.1
to get the value of $Q$. $C$ is the result of iRLC.
This produces $D$ which is the DCT coefficient matrix.
\[
D_{i,j} = Q_{i,j} * C_{i,j}
\]

It's just multiplication.

\subsubsection{Inverse 2D DCT (iDCT)}

$T$ and $T'$ are the same from section 2.3.
$D$ is the result of De-Quantization.
$N$ is our resultant \textit{N}ew 8x8 matrix block.

Equation 2:
\[
N = \text{round}(T' D T) + 128
\]
\noindent
The shift ($128$) is just to get back to a $[0,256)$ range for the values.

\subsubsection{Convert $\text{YC}_{b}\text{C}_{r}$ to RGB}
\[
\begin{bmatrix}
R \\
G \\
B

\end{bmatrix}
=
\Bigg{(}
\begin{bmatrix}
Y \\
C_{b} \\
C_{r}
\end{bmatrix}
+
\begin{bmatrix}
0 \\
-128 \\
-128
\end{bmatrix}
\Bigg{)}
\begin{bmatrix}
1 & 0 & 1.402 \\
1 & -0.34414 & -0.71414 \\
1 & 1.772 & 0
\end{bmatrix}
\]
Simplifies to this:
\begin{align*}
R & = & Y & & + 1.402 (C_{r} - 128) \\
G & = & Y & - 0.34414 (C_{b} - 128) & - 0.71414 (C_{r} - 128) \\
B & = & Y & + 1.772 (C_{b} - 128) &
\end{align*}

\noindent
Annnnd that's it.

\subsection{Error Computation}

This is trivial.  Just including the equations for completeness.

\subsubsection{Error}
$I$ is the original image.
$I'$ is the result of compression and then decompression.
\[
MSE = \frac{1}{MN}\sum\limits_{y=1}^{M}\sum\limits_{x=1}^N
\bigg{[}
I(x,y) - I'(x,y)
\bigg{]}^2
\]

\subsubsection{Peak Signal-to-Noise Ratio}
\[
PSNR = 20 \text{log}_{10}\bigg{(}\frac{255}{\sqrt{MSE}}\bigg{)}
\]

\subsection{Display}

We do not need a graphical user interface.
Command line is fine as long as it is well-defined.  We can discuss this later.
\\\\
Basically, we need to dump internal-states during processing for:
\begin{itemize}
\item Error and PSNR for Y, Cb, Cr, R, G, and B
\item DCT Coefficient matrices for all blocks
\end{itemize}

\noindent
Then probably just write a small application to process and display the dumps
in a pretty format.

\section{Meeting 1 Notes (March 20)}

\noindent
Attendies: Christy, Josh, and Paul.


\noindent
Main topic was tasking.  Josh led the discussion.
\begin{enumerate}
\item Christy:
\begin{enumerate}
\item RGB $<->$ YCbCr conversions
\item Chroma Subsampling
\end{enumerate}
\item Josh:
\begin{enumerate}
\item RLC and inverse
\item DCT and inverse
\item Quantization and inverse
\item Image Loop (splitting into 8x8 chunks, etc.)
\end{enumerate}
\item Paul:
\begin{enumerate}
\item Make Github Repository
\item Error Computation
\item CLI
\end{enumerate}
\end{enumerate}

\section{Meeting 2 Notes (March 27)}
\noindent
Attendies: Christy and Josh.


\noindent
Topic was status and next steps.


\noindent
Status:
\begin{enumerate}
\item Christy:
\begin{enumerate}
\item \sout{RGB $<->$ YCbCr conversions} (DONE)
\item \sout{Chroma Subsampling} (DONE)
\end{enumerate}
\item Josh:
\begin{enumerate}
\item \sout{RLC and inverse} (DONE)
\item \sout{DCT and inverse} (DONE)
\item \sout{Quantization and inverse} (DONE)
\item \sout{Image Loop (splitting into 8x8 chunks, etc.)} (DONE)
\end{enumerate}
\item Paul:
\begin{enumerate}
\item \sout{Make Github Repository} (DONE)
\item \sout{Error Computation} (DONE)
\item CLI (PARTIAL)
\end{enumerate}
\end{enumerate}

\noindent
Josh ended up completing the CLI so that Paul could work on Senior Design.


\end{document}
